% !TeX TS-program = xelatex
\documentclass[aspectratio=169, 12pt, xcolor=table]{beamer}
\usefonttheme{professionalfonts}
\usefonttheme{serif}
\usepackage[T1]{fontenc}
\usepackage{fontspec-xetex}

\usepackage{tikz}

\usepackage{booktabs}
\usepackage{ifthen}
\usepackage{listings}
\usepackage{subcaption}

\usetikzlibrary{shapes.geometric, arrows}

\setmainfont{Lato}

%\PassOptionsToPackage[more=table]{xcolor}

% Local configuration
\renewcommand{\figurename}{}
\DeclareCaptionFormat{custom}
{%
	\tiny #3
}
\captionsetup{format=custom}

% Title stuff
\title{Programming}
\subtitle{More Loops}
\date{Week 03}
\author{Vassilis Markos, Mediterranean College}

\usetheme{streamline}

% Local Commands
\newcommand{\ohref}[1]{\href{#1}{\texttt{#1}}}

% Code listings

\definecolor{codegreen}{rgb}{0,0.6,0}
\definecolor{codegray}{rgb}{0.5,0.5,0.5}
\definecolor{codepurple}{rgb}{0.58,0,0.82}
\definecolor{backcolour}{rgb}{0.95,0.95,0.92}

\lstdefinestyle{mystyle}{
	backgroundcolor=\color{backcolour},   
	commentstyle=\color{codegreen},
	keywordstyle=\color{magenta},
	numberstyle=\tiny\color{codegray},
	stringstyle=\color{codepurple},
	basicstyle=\ttfamily\footnotesize,
	breakatwhitespace=false,         
	breaklines=true,                 
	captionpos=b,                    
	keepspaces=true,                 
	numbers=left,                    
	numbersep=5pt,                  
	showspaces=false,                
	showstringspaces=false,
	showtabs=false,                  
	tabsize=2
}

\lstset{style=mystyle}

% Tikz style

\tikzstyle{startstop} = [ellipse, rounded corners, minimum width=2cm, minimum height=0.8cm, text centered, draw=black, fill=none]
\tikzstyle{io} = [trapezium, trapezium left angle=70, trapezium right angle=110, minimum width=2cm, minimum height=0.8cm, text centered, draw=black, fill=none]
\tikzstyle{process} = [rectangle, minimum width=2cm, minimum height=0.8cm, text centered, draw=black, fill=none]
\tikzstyle{decision} = [diamond, minimum width=2cm, minimum height=0.8cm, text centered, draw=black, fill=none]
\tikzstyle{arrow} = [thick,->,>=stealth]

% makeatletter stuff

\makeatletter
\newcommand{\arabicthree}[1]{\expandafter\@arabicthree\csname c@#1\endcsname}
\newcommand{\@arabicthree}[1]{\ifnum #1<100 0\fi\ifnum #1<10 0\fi\number#1}
\makeatother

\newcounter{exno}
\setcounter{exno}{0}

\newcommand{\exno}{\stepcounter{exno}In--class Exercise \#\arabicthree{exno}}

\begin{document}

	\begin{frame}
		\titlepage
	\end{frame}

	\begin{frame}{Contents}
		\tableofcontents
	\end{frame}

%	\section{A Brief Intro}\label{sec:a-brief-intro}
%	
%	\sectionframe
%	
%	\begin{headsup}{Desperate Times, Desperate Measures}
%		\begin{minipage}[t]{0.40\textwidth}
%			\vspace{0pt}
%			Since we are transitioning on a new platform, things are still a bit quirky. So, to make sure we keep track of who is here and who is not, please \textbf{scan the QR code shown next} or \textbf{click the link below it} and fill in this form with your information (confidential).
%		\end{minipage}\hfill
%		\begin{minipage}[t]{0.58\textwidth}
%			\vspace{0pt}
%			\raggedleft
%			\includegraphics[scale=0.35]{../../assets/attendance_form.png}
%			\centering
%			\ohref{https://forms.gle/4yiYhonrjuVs4sCv6}
%		\end{minipage}
%	\end{headsup}
%
%	\begin{headsup}{Desperate Times, Desperate Measures}
%		\begin{minipage}[t]{0.50\textwidth}
%			\vspace{0pt}
%			For similar reasons, we will also be using a public shared drive folder to keep our materials as long as our platform is a bit unstable. To visit the platform and download this lecture's materials please use the QR shown right or the link below.
%		\end{minipage}\hfill
%		\begin{minipage}[t]{0.48\textwidth}
%			\vspace{0pt}
%			\raggedleft
%			\includegraphics[scale=0.25]{../../assets/shared_folder.png}
%			\centering
%		\end{minipage}
%		\vfill
%		\ohref{https://shorturl.at/FVszq}
%		\begin{scriptsize}
%			\ohref{https://drive.google.com/drive/folders/1cUY\_XNJLyGNHRRgfJbxXemf8AFgiMnaF?usp=sharing}
%		\end{scriptsize}
%	\end{headsup}

	\section{Last Time's Exercises}\label{sec:last-time-s-exercises}
	
	\sectionframe
	
	\begin{frame}{\exno}
		A pyflix subscription (fictional python tutorials streaming service) has the following pricing scheme:
		\begin{itemize}
			\item The first 5 tutorials are free.
			\item The next 10 tutorials cost 5\$ each.
			\item The next 20 tutorials cost 4\$ each.
			\item Any further tutorials cost 2.5\$ each.
		\end{itemize}
		Write a Python program that asks the user the number of tutorials they want to attend and computes the corresponding total cost.
	\end{frame}
	
	\begin{frame}{\exno}
		Rewrite the following without any \texttt{elif}, i.e., only using \texttt{if-else}:
		\lstinputlisting[language={Python}]{../source/exercise_002.py}
	\end{frame}
	
	\begin{frame}{\exno}
		A quadratic equation, $ax^2+bx+c=0$, $a\neq0$, is solved using the following formula:
		\[x=\frac{-b\pm\sqrt{\Delta}}{2a},\quad \Delta=b^2-4ac.\]
		if $\Delta>0$. If $\Delta=0$ then the equation has a single solution, $x=-\frac{b}{2a}$, while if $\Delta<0$ the equation has no (real) roots.
		
		Write a Python program that asks the user to provide the three coefficients of a quadratic equation, $a,b,c$, and prints its solution(s), if any, or an appropriate message if no solutions exist.
	\end{frame}
	
	\begin{frame}{\exno}
		Assuming that you are allowed to use only division by (and modulo of) 2 and 3, write a python program that:
		\begin{itemize}
			\item Asks the user for an integer number, $n$.
			\item Prints on screen an appropriate message for all the following cases:
			\begin{itemize}
				\item Whether $n$ is a multiple of 2.
				\item Whether $n$ is a multiple of 3.
				\item Whether $n$ is a multiple of 6.
				\item Whether $n$ is a multiple of 24.
			\end{itemize}
			\item Explain your rationale by providing appropriate comments in your code.
		\end{itemize}
	\end{frame}
	
	\begin{frame}{\exno}
		A number $n$ divides a number $m$ if \texttt{m \% n == 0}. Write a Python program that asks the user for a positive integer $n$ and prints on screen all of its divisors.
		
		Your program \textbf{should check that the user indeed provides a positive integer}, i.e., it should ask the user to provide another number in case they provided a non--positive integer (ignore cases where user input might not be integer).
	\end{frame}
	
	\begin{frame}{\exno}
		\begin{itemize}
			\item We say that $n$ is a proper divisor of $m$ if $n$ is a divisor of $m$ and $n\neq m$. A positive integer is said to be \textbf{perfect} if it is equal to the sum of its proper divisors.
			\item Write a Python program that asks the user for a positive integer and checks if its is perfect or not, printing a relevant message on screen.
			\item Your program should check that the user indeed provides a positive integer as in the previous exercise.
		\end{itemize}
	\end{frame}
	
	\begin{frame}{\exno}
		\begin{itemize}
			\item Two numbers are said to be \textbf{amicable} if each one is equal the sum of the proper divisors of the other.
			\item Write a Python program that asks the user for two positive integers and checks if they are amicable or not, printing a relevant message on screen.
			\item Your program should check that the user indeed provides a positive integer as in the previous exercise.
		\end{itemize}
	\end{frame}

	\section{Loops (Revisited)}\label{sec:loops--revisited}
	
	\sectionframe
	
	\begin{frame}{A Simple Program}
		What does the following Python program do?
		\lstinputlisting[language={Python}]{../source/loops_001.py}
	\end{frame}

	\begin{frame}{The \texttt{for} Loop}
		Python also offers another way to loop through things:
		\lstinputlisting[language={Python}]{../source/loops_002.py}
	\end{frame}

	\begin{frame}[fragile]{The \texttt{for} Loop}
		The general syntax of a \texttt{for} loop is:
		\begin{verbatim}
for <variable> in <iterable>:
    <stuff to be repeated>
\end{verbatim}
		\begin{itemize}
			\item In Python, iterables are things we can iterate through, which, for now, should more or less be \texttt{range}s.
			\item But, what is a \texttt{range}?
		\end{itemize}
	\end{frame}

	\begin{frame}{What Is A \texttt{range}?}
		The easiest way to learn about things in Python is to print them. So, what does the following print?
		\lstinputlisting[language={Python}]{../source/loops_003.py}
	\end{frame}

	\begin{frame}{Ranges}
		In Python, ranges are:
		\begin{itemize}
			\item Objects (as everything) that represent a range of values.
			\item We can think of them as lists of values, but they are not actually lists, as they \textbf{generate} the next element upon request.
			\item We can use ranges to iterate over them using \texttt{for} loops, as we have just seen.
		\end{itemize}
	\end{frame}

	\begin{frame}{\texttt{range} Syntax}
		There are three ways to declare a range:
		\begin{itemize}
			\item \texttt{range(n)} creates a range with integer values from \texttt{0} (included) to \texttt{n} (excluded).
			\item \texttt{range(m, n)} creates a range with integer values from \texttt{m} (included) to \texttt{n} (excluded). If $m \geq n$ then an empty range is produced.
			\item \texttt{range(m, n, s)} creates a range with integer values from \texttt{m} (included) to \texttt{n} (excluded) with a step of \texttt{s}, i.e., it skips every \texttt{s} values. Again, $m\geq n$ results to an empty range for $s>0$ while it works well for $s<0$.
		\end{itemize}
	\end{frame}

	\begin{frame}{What Will This Print?}
		\lstinputlisting[language={Python}]{../source/loops_004.py}\pause
		\begin{itemize}
			\item It prints the squares of numbers from $0$ to $n-1$.\pause
			\item How can we make this prints the squares of numbers from $1$ to $n$ using a \texttt{for} loop?
		\end{itemize}
	\end{frame}

	\begin{frame}{An Easy Fix}
		One way could be as follows:
		\lstinputlisting[language={Python}]{../source/loops_005.py}\pause
		How can we print the results in \textbf{reverse order?}
	\end{frame}

	\begin{frame}{In Reverse Order}
		Remember, ranges can take a negative step:
		\lstinputlisting[language={Python}]{../source/loops_006.py}\pause
		\textbf{Be careful!} The starting point of a range is included while the ending is excluded!
	\end{frame}

	\begin{frame}[fragile]{A \texttt{print()} Interlude}
		So far, the outputs of the above programs should look like the one below:
		\begin{verbatim}
Please, enter an integer: 5
25
16
9
4
1
\end{verbatim}
		While correct, this is a bit ugly, as numbers are printed on separate rows
	\end{frame}

	\begin{frame}{A \texttt{print()} Interlude}
		Python's \texttt{print()} allows us to modify its behaviour through some keyword arguments (we shall explain deeply what a keyword argument is in upcoming lectures):
		\begin{itemize}
			\item The \texttt{end} argument allows us to determine what will be the last character appended after everything has been printed on screen. It defaults to a newline character.
			\item The \texttt{sep} argument allows us to determine which character is printed between the arguments provided to \texttt{print()} for printing. It defaults to a single space.
		\end{itemize}
	\end{frame}

	\begin{frame}{A \texttt{print()} Interlude}
		So, for instance, we can modify our last program as follows, to print all numbers on the same line:
		\lstinputlisting[language={Python}]{../source/loops_007.py}\pause
		There are some formatting issues, though. How can we address them?
	\end{frame}

	\begin{frame}{A \texttt{print()} Interlude}
		To begin with, we can add an empty print, which, by default, prints just a newline, so as to change line once we have print all numbers, as follows:
		\lstinputlisting[language={Python}]{../source/loops_008.py}\pause
		What about the trailing comma after the last printed number?
	\end{frame}

	\begin{frame}{A \texttt{print()} Interlude}
		We can capture this case using an \texttt{if} as follows:
		\lstinputlisting[language={Python}]{../source/loops_009.py}
	\end{frame}

	\begin{frame}{Python's Inline \texttt{if-elif-else}}
		Python also offers a way to write simple \texttt{if} statements in a single line (much like the ternary operator in other languages):
		\lstinputlisting[language={Python}]{../source/loops_010.py}
	\end{frame}

	\begin{frame}{What Will This Print?}
		\lstinputlisting[language={Python}]{../source/loops_011.py}\pause
		It prints the following sum:
		\[s=\frac{1}{1^2}+\frac{1}{2^2}+\frac{1}{3^2}+\frac{1}{4^2}+\cdots+\frac{1}{n^2}.\]
	\end{frame}

	\begin{frame}{Odd Only?}
		How can we compute the sum of the odd terms, only, i.e.:
		\[s=\frac{1}{1^2}+\frac{1}{3^2}+\frac{1}{5^2}+\frac{1}{7^2}+\cdots\]\pause
		\lstinputlisting[language={Python}]{../source/loops_012.py}
	\end{frame}
	
	\begin{frame}{Even Only?}
		How can we compute the sum of the even terms, only, i.e.:
		\[s=\frac{1}{2^2}+\frac{1}{4^2}+\frac{1}{6^2}+\frac{1}{8^2}+\cdots\]\pause
		\lstinputlisting[language={Python}]{../source/loops_013.py}
	\end{frame}

	\begin{frame}{\texttt{while} Loops To \texttt{for} Loops}
		Write the following program using no \texttt{while} loops:
		\lstinputlisting[language={Python}]{../source/loops_015.py}
	\end{frame}

	\begin{frame}{\texttt{while} Loops To \texttt{for} Loops}
		One solution could be as follows:
		\lstinputlisting[language={Python}]{../source/loops_016.py}
	\end{frame}

	\begin{frame}{\texttt{while} Loops To \texttt{for} Loops}
		Write the following program using no \texttt{while} loops:
		\lstinputlisting[language={Python}]{../source/loops_017.py}
	\end{frame}

	\begin{frame}{\texttt{while} Loops To \texttt{for} Loops}
		One solution could be as follows:
		\lstinputlisting[language={Python}]{../source/loops_018.py}
	\end{frame}

	\begin{frame}{\texttt{break}}
		In Python, \texttt{break} is used to stop the execution of a loop and break out of it.
		\begin{itemize}
			\item All code in the same block and below a \texttt{break} won't be executed.
			\item Using \texttt{break} we can write any code we would write using \texttt{while}, i.e., capture more complex termination conditions.
			\item As above, observe that some times the range limits when transitioning from \texttt{while} to \texttt{for} loops have to be carefully adjusted!
		\end{itemize}
	\end{frame}

	\begin{frame}{What Will This Print?}
		\lstinputlisting[language={Python}]{../source/loops_014.py}
	\end{frame}

	\begin{frame}[fragile]{Printing Grids}
		For inputs 4 and 6 this should print something like the following:
		\begin{verbatim}
0-0 0-1 0-2 0-3 0-4 0-5 
1-0 1-1 1-2 1-3 1-4 1-5 
2-0 2-1 2-2 2-3 2-4 2-5 
3-0 3-1 3-2 3-3 3-4 3-5
\end{verbatim}
		Can you explain why?
	\end{frame}

	\begin{frame}[fragile]{Nested Loops}
		Nesting loops results to the inner loops being repeated so many times as the external loop dictates. For instance:
		\begin{lstlisting}[language={Python}]
for i in range(10):
    for j in range(20):
        <stuff>
\end{lstlisting}
		will repeat \texttt{<stuff>} $10\times 20 = 200$ times, since:
		\begin{itemize}
			\item the inner loop repeats \texttt{<stuff>} 20 times, and;
			\item the outer loop repeats the inner loop 10 times.
		\end{itemize}
		So, it is important to remember that nesting loops leads to \textbf{multiplicatively longer} execution times!
	\end{frame}
	
	\section{Fun Time!}\label{sec:fun-time}
	
	\sectionframe
	
	\setcounter{exno}{0}
	
	\begin{frame}{\exno}
		Write a Python program that computes the following sum:
		\[s=1+\frac{1}{2}+\frac{1}{4}+\frac{1}{8}+\frac{1}{16}+\cdots+\frac{1}{2^n},\]
		where $n$ is provided by the user.
		
		Then, make any modifications needed to compute the following sum:
		\[s=1-\frac{1}{2}+\frac{1}{4}-\frac{1}{8}+\frac{1}{16}-\cdots\pm\frac{1}{2^n}.\]
	\end{frame}

	\begin{frame}{\exno}
		Write a Python program that asks the user for a positive integer, $n$, and computes its factorial, denoted by $n!$. As a reminder, the factorial of a number, $n$, is given by the following formula:
		\[n!=n\cdot(n-1)\cdot(n-2)\cdots 2\cdot 1.\]
		So, for instance:
		\begin{align*}
			3!&=3\cdot2\cdot1=6,\\
			5!&=5\cdot4\cdot3\cdot2\cdot1=120,\\
			8!&=8\cdot7\cdot6\cdot5\cdot4\cdot3\cdot2\cdot1=40320.
		\end{align*}
	\end{frame}

	\begin{frame}{\exno: Part A}
		In combinatorics, a quite useful quantity is the number of subsets of $k$ elements from a set of $n$ elements, with $k<n$, given by the following formula:
		\[\binom{n}{k}=\frac{n!}{k!(n-k)!}.\]
		Using code from your previous exercise, write a Python program that asks the user for $n, k$ and computes $\binom{n}{k}$ (which is often called the \textbf{binomial coefficient}).
	\end{frame}

	\addtocounter{exno}{-1}
	
	\begin{frame}{\exno: Part B}
		Test your previous program on the following input:
		\[n=10000,\quad k=5000.\]
		\vspace{-\topsep}
		\begin{itemize}
			\item How did it perform?
			\item Can you find a way to fix it? You can, of course, look around the web for ideas.
			\item Explain the rationale of your solution.
		\end{itemize}
	\end{frame}

	\begin{frame}{\exno: Part A}
		A positive integer number, $n$, is said to be prime if:
		\begin{itemize}
			\item $n>1$, and;
			\item the only divisors of $n$ are $1$ and $n$.
		\end{itemize}
		Explain why the following program, intended to check whether a number $n$ is prime is wrong:
		\lstinputlisting[language={Python}, basicstyle=\ttfamily\scriptsize]{../source/exercise_004a.py}
	\end{frame}
	
	\addtocounter{exno}{-1}
	
	\begin{frame}{\exno: Part B}
		A positive integer number, $n$, is said to be prime if:
		\begin{itemize}
			\item $n>1$, and;
			\item the only divisors of $n$ are $1$ and $n$.
		\end{itemize}
		Write a Python program that asks the user for a positive integer and computes whether it is prime or not. Test your program on the following inputs:
		\[1, 2, 3, 4, 5, 7, 9, 14, 23, 57, 101.\]
		The outputs should be (\texttt{True} for prime, \texttt{False} for non--prime):
		\begin{center}
			\texttt{False}, \texttt{True}, \texttt{True}, \texttt{False}, \texttt{True}, \texttt{True}, \texttt{False}, \texttt{False}, \texttt{True}, \texttt{False}, \texttt{True}.
		\end{center}
	\end{frame}

	\addtocounter{exno}{-1}
	
	\begin{frame}{\exno: Part C}
		Test your previous program about primes on the following input:
		\[1234567891.\]
		\vspace{-\topsep}
		\begin{itemize}
			\item How long did it take to terminate?
			\item Can you make it run faster?
			\item How long does your new implementation take on $10101012107$?
		\end{itemize}
	\end{frame}

	\begin{frame}{\exno}
		\begin{minipage}{0.5\textwidth}
			Write a Python function that asks the user for the number of rows and columns and prints a shape like what is shown next on screen (this is for 5 rows and 6 columns).
		\end{minipage}\hfill
		\begin{minipage}{0.45\textwidth}
			\centering
			\begin{tikzpicture}
				\foreach \i in {0,...,5} {
					\foreach \j in {0,...,4} {
						\node (\i\j) at (\i, \j) {\texttt{*}};
					}
				}
			\end{tikzpicture}
		\end{minipage}
	\end{frame}

	\begin{frame}{\exno}
		\begin{minipage}{0.5\textwidth}
			Write a Python function that asks the user for the number of rows / columns (it should be the same in this case) and prints a shape like what is shown next on screen (this is for 5 rows / columns).
		\end{minipage}\hfill
		\begin{minipage}{0.45\textwidth}
			\centering
			\begin{tikzpicture}
			\foreach \i in {0,...,4} {
				\foreach \j in {0,...,4} {
					\ifthenelse{\i > \j}{}{
						\node (\i\j) at (\i, \j) {\texttt{*}};
					}
				}
			}
			\end{tikzpicture}
		\end{minipage}
	\end{frame}

	\begin{frame}{\exno}
		\begin{minipage}{0.5\textwidth}
			Write a Python function that asks the user for the number of rows / columns (it should be the same in this case) and prints a shape like what is shown next on screen (this is for 5 rows / columns).
		\end{minipage}\hfill
		\begin{minipage}{0.45\textwidth}
			\centering
			\begin{tikzpicture}
			\foreach \i in {0,...,4} {
				\foreach \j in {0,...,4} {
					\ifthenelse{\i < \j}{}{
						\node (\i\j) at (\i, \j) {\texttt{*}};
					}
				}
			}
			\end{tikzpicture}
		\end{minipage}
	\end{frame}
	
	\begin{frame}{\exno}
		\begin{minipage}{0.5\textwidth}
			Write a Python function that asks the user for the number of rows / columns (it should be the same in this case) and prints a shape like what is shown next on screen (this is for 5 rows / columns).
		\end{minipage}\hfill
		\begin{minipage}{0.45\textwidth}
			\centering
			\begin{tikzpicture}
			\foreach \i in {0,...,4} {
				\foreach \j in {0,...,4} {
					\ifthenelse{\i=0\OR \j=0}{
						\node (\i\j) at (\i, \j) {\texttt{*}};
					}{}
				}
			}
			\end{tikzpicture}
		\end{minipage}
	\end{frame}

	\begin{frame}{\exno}
		\begin{minipage}{0.5\textwidth}
			Write a Python function that asks the user for the number of rows / columns (it should be the same in this case) and prints a shape like what is shown next on screen (this is for 5 rows / columns).
		\end{minipage}\hfill
		\begin{minipage}{0.45\textwidth}
			\centering
			\begin{tikzpicture}
			\foreach \i in {0,...,4} {
				\foreach \j in {0,...,4} {
					\ifthenelse{\i=0\OR \j=0 \OR \i=4\OR \j=4}{
						\node (\i\j) at (\i, \j) {\texttt{*}};
					}{}
				}
			}
			\end{tikzpicture}
		\end{minipage}
	\end{frame}
	
	\begin{frame}{\exno}
		Start working on all Labs found in today's materials \texttt{homework} directory.
		
		To help me asses those files, you can name them as follows:
		\begin{center}
			\texttt{task\_xxx.py}
		\end{center}
		where \texttt{xxx} is the number of the task. For instance, task 5 file could be named \texttt{task\_005.py}.
		
		Submit your work via email at: \texttt{v.markos@mc-class.gr}
	\end{frame}

	\begin{frame}{Homework}
		\begin{itemize}
			\item In this week's materials, under the \texttt{homework} directory, you can find some Python programming Tasks. Complete as many of them as you can (preferably all).
			\item This is important, since tasks such as those provided with this lecture's materials will most probably be part of your course assessment portfolio. So, take care to solve as many of those tasks as possible!
			\item Share your work at: \texttt{v.markos@mc-class.gr}
		\end{itemize}
	\end{frame}

	\begin{frame}{Any Questions?}
		\begin{minipage}{0.35\textwidth}
			\raggedright
			Do not forget to fill in the questionnaire shown right!
		\end{minipage}\hfill
		\begin{minipage}{0.58\textwidth}
			\vspace{0pt}
			\raggedleft
			\includegraphics[scale=0.4]{../../assets/post_lesson_assessment.png}
			\centering
			\ohref{https://forms.gle/dKSrmE1VRVWqxBGZA}
		\end{minipage}
	\end{frame}
	
\end{document}